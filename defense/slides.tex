\documentclass[12pt,dvipsnames]{beamer}
\geometry{paper=a6paper,landscape}

\usetheme{default}% I recommend
\usepackage[utf8]{inputenc}
\usepackage{utopia}
\usecolortheme{seahorse}
% \or\usetheme{Singapore}
% \or\usetheme{Boadilla}
% \or\usetheme{Pittsburgh}
% \or\usetheme{Madrid}
% \or\usetheme{Warsaw} % common choice, but often poor
% \fi

\usepackage{xcolor}
\usepackage{listings}
\usepackage{semantic}
\usepackage{graphicx,pgfplots,parskip}
\usepackage[utf8]{inputenc}
\usepackage[T1]{fontenc}
\usepackage{microtype}
\usepackage{amsmath}
\usepackage{amssymb}
\usepackage{ifthen}
\newcommand\hmmax{0} % default 3
% \newcommand\bmmax{0} % default 4
% \usepackage{bm}
\usepackage{stmaryrd}
\usepackage{proof}
\usepackage{graphicx}
\usepackage[export]{adjustbox}
\usepackage{listings}
\usepackage{tikz}
\usepackage{semantic}
\usepackage{setspace}
\usepackage{wrapfig}
\usepackage{caption}
\usepackage{subcaption}
\usepackage{esvect}
\usepackage{tikz}
\usepackage{booktabs} %% http://ctan.org/pkg/booktabs
\usepackage{wasysym}
\usepackage{array,multirow}
\usepackage{balance}
\usepackage{minted}
\usemintedstyle{tango}
\usepackage[backend=bibtex, style=authoryear-comp]{biblatex}
%%%%%%%%%%%%%%%%%%%%% MACROS %%%%%%%%%%%%%%%%%%%%%%%%%%%


\definecolor{lightgray}{gray}{0.9}
%\newcommand{\highlight}[1]{\colorbox{lightgray}{$\displaystyle #1$}}
\newcommand{\hi}[1]{\colorbox{lightgray}{$\displaystyle #1$}}

\newcommand{\tagsc}[1]{\tag{\textsc{#1}}}
\newcommand{\newname}[2]{\newcommand{#1}{\ensuremath{#2}}}

\newcommand{\overbar}[1]{\mkern 1.5mu\overline{\mkern-1.5mu#1\mkern-1.5mu}\mkern 1.5mu}

\newcommand{\size}{size}
\newcommand{\cast}{\Rightarrow}
\newcommand{\CoercionTyping}{\Longrightarrow}
\newcommand{\castE}[4][]{\ensuremathc{#2 : #3 \Rightarrow^{#1} #4}}

\newcommand{\key}[1]{\texttt{#1}}

% Types
\newcommand{\UnitT}{\ensuremath{\mathsf{()}}}
\newcommand{\IntT}{\ensuremath{\mathsf{Int}}}
\newcommand{\BoolT}{\ensuremath{\mathsf{Bool}}}
\newcommand{\DynT}{\ensuremath{\mathsf{\star}}}
\newcommand{\RefT}[1]{\ensuremath{\mathsf{Ref}\ #1}}
\newcommand{\RefPT}[1]{\ensuremath{\mathsf{Ref}\ #1}}
%\newcommand{\RefMT}[1]{\ensuremath{\mathsf{Ref}_m #1}}
\newcommand{\FunT}[2]{\ensuremath{#1 \to #2}}
\newcommand{\CFunT}[3]{\ensuremath{(\overbar{#1},#2) \cast #3}}
\newcommand{\PairT}[2]{\ensuremath{#1 \times #2}}
\newcommand{\Unit}{\ensuremath{\text{Unit}}}
\newcommand{\CClosT}[2]{\ensuremath{#1 \to #2}}
\newcommand{\CoT}[2]{\ensuremath{#1 \Longrightarrow #2}}

\newcommand{\IntC}{\ensuremath{\mathtt{Int}}}
\newcommand{\BoolC}{\ensuremath{\mathtt{Bool}}}
\newcommand{\DynC}{\ensuremath{\mathtt{Dyn}}}
\newcommand{\RefC}[1]{\ensuremath{\mathtt{Ref}\ #1}}

% Relations

\newcommand{\TypePreciseness}{\sqsubseteq}
\newcommand{\ExprTypePreciseness}{\sqsubseteq}
\newcommand{\StoreTypePreciseness}{\sqsubseteq_p}
\newcommand{\StoreTypeExtension}{\sqsubseteq_e}
\newcommand{\StoreTypeProgress}{\sqsubseteq_{p/e}}
\newcommand{\static}{\Rbag}
\newcommand{\ShallowConsist}{\small\smile}

% Coercions
\newcommand{\idC}{\ensuremath{\mathtt{\iota}}}
\newcommand{\InjectCoercion}[1]{\ensuremath{#1!}}
\newcommand{\seqInjC}[2]{\ensuremath{\seqC{#1}{\InjectCoercion{#2}}}}
\newcommand{\ProjectCoercion}[1]{\ensuremath{#1?}}
\newcommand{\seqPrjC}[2]{\ensuremath{\seqC{\ProjectCoercion{#1}}{#2}}}
\newcommand{\FunctionCoercion}[2]{\ensuremath{#1 \to #2}}
\newcommand{\RefCoercion}[1]{\ensuremath{\mathsf{Ref}\ #1}}
\newcommand{\seqC}[2]{\ensuremath{#1 \, ; \, #2}}
\newcommand{\FailCoercion}[2]{\ensuremath{\bot^{#1,#2}}}
\newcommand{\FailCoercionSE}{\ensuremath{\bot}}
\newcommand{\PairCoercion}[2]{\ensuremath{#1 \times #2}}

\newcommand{\coercion}[1]{\ensuremath{\langle #1 \rangle}}
\newcommand{\coerce}[2]{\ensuremath{#1 \coercion{#2}}}
\newcommand{\coerced}[2]{\ensuremath{\coerce{#1}{#2}}}
\newcommand{\app}[2]{\ensuremath{#1 \, #2}}
\newcommand{\lam}[2]{\ensuremath{\lambda #1 .\, #2}}
\newcommand{\blame}[1]{\ensuremath{\mathtt{blame}\, #1}}
\newcommand{\mkRef}[1]{\ensuremath{\mathtt{ref}\ #1}}
\newcommand{\allocT}[2]{\ensuremath{\mathtt{ref}\ #1 @ #2}}
\newcommand{\alloc}[1]{\allocT{#1}{\AllTa}}
\newcommand{\deref}[1]{\ensuremath{\mathtt{!} #1}}
\newcommand{\derefT}[1]{\ensuremath{\mathtt{!} #1 @ \AllTa}}
\newcommand{\DerefT}[2]{\ensuremath{\mathtt{!} #1 @ #2}}
\newcommand{\setref}[2]{\ensuremath{#1\ {:=}\ #2}}
\newcommand{\setrefT}[2]{\ensuremath{#1\ {:=}\ #2 @ \AllTa}}
\newcommand{\SetrefT}[3]{\ensuremath{#1\ {:=}\ #2 @ #3}}
\newcommand{\error}{\ensuremath{\mathtt{error}}}
\newcommand{\Error}{\ensuremath{\mathtt{Err}}}
\newcommand{\dom}[1]{\ensuremath{\text{dom}(#1)}}
\newcommand{\pair}[2]{\ensuremath{(#1,#2)}}
\newcommand{\fst}[1]{\ensuremath{(\mathtt{fst}\, #1)}}
\newcommand{\snd}[1]{\ensuremath{(\mathtt{snd}\, #1)}}

\newcommand{\glbt}{\sqcap}

\newcommand{\var}[1][x]{\ensuremath{#1}}
\newcommand{\expr}[1][e]{\ensuremath{#1}}
\newcommand{\const}{\ensuremath{k}}

\newcommand{\uncoerced}{\ensuremath{u}}
\newcommand{\val}{\ensuremath{v}}
\newcommand{\DelayedCastA}{\ensuremath{dc}}
\newcommand{\DelayedCastB}{\ensuremath{dc'}}
\newcommand{\DelayedCastC}{\ensuremath{dc''}}
\newcommand{\ReducibleDelayedCast}{\ensuremath{dc^{+}}}

\newcommand{\hole}{\ensuremath{\square}}
\newcommand{\context}[2]{\ensuremath{ #1 [ #2 ] }}

\newcommand{\InertCoercion}[1]{\ensuremath{#1\uparrow}}
\newcommand{\InertCoercionA}{\InertCoercion{\seCa}}
\newcommand{\InertCoercionB}{\InertCoercion{\seCb}}
\newcommand{\InertFinalCoercion}{\ensuremath{\seFinalC\uparrow}}
\newcommand{\InertMiddleCoercion}{\ensuremath{\seInterC\uparrow}}
\newcommand{\ActiveCoercion}[1]{\ensuremath{#1\downarrow}}
\newcommand{\ActiveCoercionA}{\ActiveCoercion{\seCa}}
\newcommand{\ActiveCoercionB}{\ActiveCoercion{\seCb}}
\newcommand{\ActiveFinalCoercion}{\ensuremath{\seFinalC\downarrow}}
\newcommand{\ActiveMiddleCoercion}{\ensuremath{\seInterC\downarrow}}

\newcommand{\CastCong}{\varphi}
\newcommand{\AllowCastCong}{\top}
\newcommand{\DisallowCastCong}{\bot}
\newcommand{\PureSymbol}{\obslash}
\newcommand{\ImpureSymbol}{\ocircle}
\newcommand{\PureCastReduce}{\longrightarrow_{c}^{\PureSymbol}}
% \newcommand{\PureCastDelayedCastReduce}{\longrightarrow_{dc}^{\PureSymbol}}
\newcommand{\PureCastDelayedCastReduce}{\longrightarrow}
\newcommand{\MonoCastReduce}{\longrightarrow_{c}^{\ImpureSymbol}}
\newcommand{\CastReduce}{\longrightarrow_{c}}
\newcommand{\DelayedCastReduce}{\longrightarrow_{dc}}
\newcommand{\PureReduce}{\longrightarrow_{e}^{\PureSymbol}}
\newcommand{\MonoReduce}{\longrightarrow_{e}^{\ImpureSymbol}}
\newcommand{\ProgReduce}{\longrightarrow_{e}}
\newcommand{\Lang}{\ensuremath{\textsc{GTLC}^{+}}}

% Meta Variables
\newname{\BaseT}{B}
\newname{\InjTa}{I}
\newname{\InjTb}{J}
\newname{\AllTa}{T}
\newname{\AllTb}{S}
\newname{\topC}{c}
\newname{\noFailC}{r}
\newname{\blameLabel}{p}
\newname{\addr}{a}
\newname{\seCa}{c}
\newname{\seCb}{d}
\newname{\ca}{c}
\newname{\cb}{d}
\newname{\seFinalC}{i}
\newname{\seInterC}{g}
\newname{\seIdFreeC}{f}
\newname{\CoercionT}{\mathcal{C}}
\newname{\expra}{M}
\newname{\exprb}{N}
\newname{\ctag}{w}
\newname{\uctag}{uc}
\newname{\cctag}{cc}

% Functions
\newcommand{\ComposeCoercion}{\fatsemi}
\newcommand{\ComposeMiddleCoercion}{\ensuremath{\fatsemi_{\text{\tiny g}}}}
\newcommand{\ComposeFinalCoercion}{\ensuremath{\fatsemi_i}}
\newcommand{\mkC}[2]{\ensuremath{ #1 \Rightarrow #2 }}

\newcommand{\jdgt}[2]{\ensuremath{\Gamma\vdash #1:#2}}
\newcommand{\ctit}[3]{\ensuremath{\Gamma\vdash #1 \hookrightarrow #2:#3}}
\newcommand{\mkCrcn}[2]{\ensuremath{\langle #1 \Longrightarrow #2
    \rangle}}

% Heaps
\newcommand{\store}{\ensuremath{\mu}}
\newcommand{\evstore}{\ensuremath{\nu}}
\newcommand{\storecell}[1]{\ensuremath{\addr \mapsto #1}}
\newcommand{\mstorecell}[2]{\ensuremath{\addr \mapsto #1 : #2}}
\newcommand{\StoreCell}[3]{\ensuremath{#1 \mapsto #2 : #3}}
\newcommand{\StoreValRead}{\ensuremath{\store(\addr)_{\text{val}}}}
\newcommand{\StoreRTTIRead}{\ensuremath{\store(\addr)_{\text{rtti}}}}
\newcommand{\EvStoreValRead}{\ensuremath{\evstore(\addr)_{\text{val}}}}
\newcommand{\EvStoreRTTIRead}{\ensuremath{\evstore(\addr)_{\text{rtti}}}}

\newcommand{\EvolvingStoreRTTIRead}[2]{\ensuremath{#1(#2)_{\text{rtti}}}}


\DeclareCiteCommand{\citeseries}
  {\usebibmacro{prenote}}
  {\usebibmacro{citeindex}%
    \printfield{series}}
  {\multicitedelim}
  {\usebibmacro{postnote}}

\addbibresource{all.bib}

\newcommand{\customcite}[1]{\citeauthor{#1}, \citetitle{#1},
  \citeseries{#1} \citeyear{#1}}

\newcommand{\customcitewithoutseries}[1]{\citeauthor{#1},
  \citetitle{#1}, \citeyear{#1}}

\lstdefinestyle{basic}{
%showstringspaces=false,
language=Python,
columns=fullflexible,
%basicstyle=\sffamily\small,%
basicstyle=\ttfamily,%
%columns=fixed,
%basewidth=0.49em,
%lineskip=0pt,
%escapechar=@,xleftmargin=1pc,%
keywordstyle=\ttfamily,
mathescape=true,%
moredelim=**[is][\color{blue}]{@}{@},
moredelim=[is][\color{red}]{|}{|},
moredelim=[is][\color{blue}]{~}{~},
%commentstyle=\rmfamily,%
%morekeywords={return,fix,var,proc,fun,func},%
%deletekeywords={int,bool}
}
\lstset{style=basic}

\newcommand{\lstsetgrift}{
  \lstset{%
    language=Lisp,
    numbers=left,
    stepnumber=1,
    basicstyle=\ttfamily\small,
    keywordstyle=\ttfamily\bfseries\small,
    columns=flexible,
    aboveskip=\smallskipamount,
    belowskip=\smallskipamount,
    xleftmargin=2pt,
    escapeinside={/+}{+/},
    mathescape=true,
    morekeywords=[1]{define,lambda,if,begin,letrec,let,let*,values,:,repeat,then,else,return,and,time,box,ref,unbox,box,ann},
    literate=
    {+}{{\textsf{+}}}1
    {-}{{\textsf{-}}}1
    {=>}{{$\rightarrow\;$}}2
    {lambda}{{$\boldsymbol{\lambda}$}}1
  }
}%%

\newcounter{stuff}
\newtheorem{prop}[stuff]{Proposition}

\title %optional
{Towards Efficient Gradual Typing}
 
\subtitle{via Monotonic References and Coercions}
 
\author[Almahallawi] % (optional, for multiple authors)
{Deyaaeldeen Almahallawi}
 
\institute[IU] % (optional)
{
  \inst{}%
  Doctor of Philosophy\\
  Luddy School of Informatics, Computing, and Engineering\\
  Indiana University Bloomington
}
 
\date[May 2020] % (optional)
{May 2020}

\begin{document}
\author[Almahallawi]{\begin{tabular}{r@{ }l} 
Author:      & Deyaaeldeen Almahallawi \\[1ex] 
Adviser: & Jeremy G. Siek
\end{tabular}}
\begin{frame}
\maketitle
\end{frame}

\begin{frame}{Outline}
\tableofcontents
\end{frame}

\section{Overview}

\begin{frame}{Outline}
  \tableofcontents[currentsection]
\end{frame}

\frame{
\frametitle{Gradual Typing}

\hspace{0.5in}
\begin{minipage}{0.7\textwidth}
  \center{\includegraphics[scale=.7]{yinyang.png}}
\end{minipage}

}

\begin{frame}[fragile]
\frametitle{Examples Of Gradual Typing}
\note{Gradual typing has two compelling features: soundness and interoperability}
\begin{center}
\note{Regarding soundness, programmers, compilers, and IDEs would like
  to trust their type annotations}
  \lstset {
    showstringspaces=false
  }
\begin{lstlisting}
(define r : (Ref Int) (box 42))
\end{lstlisting}
\note{this expression should be accepted}
\pause
\begin{lstlisting}
(+ 1 (unbox r))
\end{lstlisting}
\note{this expression should be rejected}
\pause
\begin{lstlisting}
(string-append (unbox r) " is forty two")
\end{lstlisting}
\pause
\note{in the absence of type annotations, casts should make sure other
  annotations are respected}
\begin{lstlisting}
(define (g y) (set-box! r y))
\end{lstlisting}
\pause
\begin{lstlisting}
(g 1)
(g "forty two")
\end{lstlisting}
\end{center}

\end{frame}

\begin{frame}[fragile]
\frametitle{Examples Of Gradual Typing (Cont'd)}
\note{Gradual typing has two compelling features: soundness and interoperability}
\begin{center}
  \lstset {
    showstringspaces=false
  }
\begin{lstlisting}
(define r1 (box 42))
(define r2 (box "forty two"))
\end{lstlisting}
\pause
\begin{lstlisting}
(define (f [x : (Ref Int)]) (+ 1 (unbox x)))
\end{lstlisting}
\pause
\begin{lstlisting}
(f r1)
\end{lstlisting}
\pause
\begin{lstlisting}
(f r2)
\end{lstlisting}
\end{center}

\end{frame}

\begin{frame}{Problem Statement And Objective}
  \begin{enumerate}
  \item Problem Statement:
  \begin{center}
    Although a lot of programmers and companies use gradually typed
    languages to write programs that is widely deployed, efficiency is a
    serious concern. The standard operational semantics for the
    Gradually Typed Lambda Calculus can exhibit unbounded space leaks
    and overheads occur even in statically typed programs.
  \end{center}
\item Objective:
  \begin{center}
    Design efficient operational semantics for gradually typed mutable
    references and vectors that could be implemented by a compiler.
  \end{center}
  \end{enumerate}
\end{frame}

% \section{Efficiency Challenges}

% \begin{frame}{Outline}
%   \tableofcontents[currentsection]
% \end{frame}

\begin{frame}[fragile]{Chains of Proxies}
    \begin{center}
    \includegraphics[scale=0.14]{quicksort.png}
  \end{center}
  % \begin{itemize}
  % \item Soundness requires performing type checking at runtime.
  % \item Type checking for higher-order values create proxies that does
  %   checking at use sites.
  % \item A chain of proxies could be created if the same value crosses
  %   multiple boundaries.
  % \end{itemize}
\end{frame}

\begin{frame}[fragile]{Dynamic Dispatch in Statically-typed Code}

\begin{lstlisting}
(define (partition [a : (Vect Int)] [p : Int] [r : Int]) : Int
    (let ([i (- p 1)]
           [x |(vector-ref a r)|])
      (repeat (j p r)
                (if (<= |(vector-ref a j)| x)
                    (begin
                      (set! i (+ i 1))
                      (swap a i j))
                    ()))
        (swap a (+ i 1) r)
        (+ i 1)))
(define (swap [a : (Vect Int)] [i : Int] [j : Int]) : Unit
    (if (= i j)
        ()
        (let ([t |(vector-ref a i)|])
          |(vector-set! a i (vector-ref a j))|
          |(vector-set! a j t)|)))
\end{lstlisting}
\end{frame}

\begin{frame}[fragile]{Solved?}
  \begin{center}
  \begin{tabular}{|c|cc|}
    \hline
    Problem/Feature & Functions & References \\
    \hline
    Chains of Proxies & \checkmark & \checkmark \\
    Dynamic Dispatch & \checkmark & \checkmark \\
    \hline
  \end{tabular}
\end{center}
\pause But, how to combine and implement them?
\end{frame}

\begin{frame}[fragile]{Solutions}
  \begin{center}
  \begin{tabular}{|c|cc|}
    \hline
    Problem/Feature & Functions & References \\
    \hline
    Chains of Proxies & Coercions\footnote[frame]{\customcite{Siek:2015ab}} & Monotonic\footnote[frame]{\customcite{Siek:2015aa}}/Coercions \\
    Dynamic Dispatch & Closure extension\footnote[frame]{\customcite{Siek:2012uq}} & Monotonic \\
    \hline
  \end{tabular}
\end{center}
\end{frame}

\begin{frame}[fragile]{Combination of Solutions}
  \begin{center}
  \begin{tabular}{|c|cc|}
    \hline
    Problem/Feature & Functions & References \\
    \hline
    Chains of Proxies & \textcolor{blue}{Coercions} & \textcolor{blue}{Monotonic}/Coercions \\
    Dynamic Dispatch & Closure extension & \textcolor{blue}{Monotonic} \\
    \hline
  \end{tabular}
\end{center}
\end{frame}

\begin{frame}{Research Questions}
  \begin{enumerate}
  \item How to integrate monotonic references into a space-efficient
    coercion calculus? (Section 5.4)\pause
  \item How to implement monotonic references in a way that only
      puts values in the heap? (Section 5.2) \pause
  \item What is the time cost of achieving space efficiency with
      coercions? (Section 8.2) % (Section~\ref{sec:cost-space-efficiency})\label{RQ:cost-space}
    \pause
  \item What is the overhead of gradual typing?
    % (Sec.~\ref{sec:external-comparison})
    I subdivide this question into the overheads on programs that are (a)
    statically typed, (b) untyped, and (c) partially
    typed. (Section 8.3) % \label{RQ:overhead}\label{RQ:static}\label{RQ:dyn}\label{RQ:partial}
    \pause
  \item Do monotonic references eliminate overheads associated with
    gradual typing from statically typed code? (Section 8.5)
    % (Section~\ref{sec:efficient-static})\label{RQ:efficient-static}
    \pause
  \item What is the overhead of using monotonic references in partially
    typed code? (Section 8.6)
    % (Section~\ref{sec:cost-efficient-static})\label{RQ:cost-efficient-static}
  \end{enumerate}
\end{frame}

\section{Literature Review}

\begin{frame}{Outline}
  \tableofcontents[currentsection]
\end{frame}

\begin{frame}{History of Integration of Static Typing and Dynamic
    Typing}
  \begin{enumerate}
  \item \customcite{Thatte:1990yv}
  \item A nominal type system for an object-oriented language with
    optional type annotations - \customcite{Anderson:2002kd}
  \item Migratory typing - \customcite{Tobin-Hochstadt:2006fk}
  \item Hybrid typing - \customcite{Gronski:2006uq}
  \item Operational semantics specification of a multi-language system -
    \customcite{Matthews:2007zr}
  \item Gradual typing for structural types - \customcite{Siek:2006bh}
  % \item Contracts for Higher-Order Functions \customcite{Findler:2002eu}
  \end{enumerate}
\end{frame}

\begin{frame}{Reviewing Coercions}
  \begin{enumerate}
  \item Designed by Henglien for compile-time optimization of dynamic
    typing - \customcite{Henglein:1994nz}
  \item Repurposed to solve the space leak problem in gradual typing -
    \customcite{Herman:2006uq}
  \item Algorithmic composition - \customcite{Siek:2015ab}
  \end{enumerate}
\end{frame}

\begin{frame}{Reviewing Performance Studies}
  \begin{enumerate}
  \item \customcite{Takikawa:2016aa}
  \item \customcite{Rastogi:2015aa}
  \item \customcite{Richards:2017aa}
  \item \customcite{Bauman:2017aa}
  \item \customcite{Muehlboeck:2017aa}
  \item \customcite{Feltey:2018aa}
  \item \customcite{10.1145/3359619.3359742}
  \end{enumerate}
\end{frame}

\begin{frame}{Reviewing Monotonic References}
  \begin{center}
    \begin{itemize}
    \item First introduced as an abstract machine in
      \customcite{Siek:2013aa} (Section 3)
    \item Reintroduced as reduction semantics by
      \customcite{Siek:2015aa} (A variant is presented in Section 4) %\footnote[frame]{\customcite{Siek:2015aa}}
    \item Values are just addresses.
    \item $\forall \addr.\ s.t.\ \addr : \RefT{\AllTa}$, 
      $\Sigma(\addr) \TypePreciseness \AllTa$
    \item Stores expressions on the heap!
    \end{itemize}
    \vspace{0.5cm}
    \includegraphics[scale=0.25]{rule.png}
  \end{center}
\end{frame}

% \begin{frame}[fragile]{Reviewing Monotonic References (Cont'd)}
%   \lstset {
%     numbers=left,
%     stepnumber=1
%   }
% \centering
% \begin{center}
% \begin{minted}
% [
% framesep=2mm,
% baselinestretch=1.2,
% bgcolor=lightgray,
% fontsize=\footnotesize,
% linenos,
% escapeinside=||
% ,mathescape=true
% ]
% {racket}
% (: f Dyn)
% (define f (lambda (x) x))
% (: r Dyn)
% (define r (box (cons f '())))
% (set-box! r (cons f r)) ;; establish a cycle
% (define (g [x : (Boxof (Pair (Dyn -> Integer)
%                              (Boxof (Pair (Integer -> Dyn)
%                                           Dyn))))])
%   ((car (unbox x)) 42))
% (g r)
% \end{minted}
% \end{center}
% \end{frame}

% \begin{frame}[fragile]{Old Work Writes Expressions to The Heap}
%   \begin{center}
%     Heap has
%     $a \mapsto \pair{\coerced{(\lambda x. x)}{\FunT{\DynT}{\DynT}
%         \cast \DynT}}{\coerced{a}{\RefT{(\PairT{\DynT}{\DynT})} \cast
%         \DynT}}$
%     \\
%     \pause
%     \vspace{0.5cm}
%     $\addr$ should be cast to $\PairT{(\FunT{\DynT}{\IntT})}{\RefT{ (\PairT{
%           (\FunT{\IntT}{\DynT}) }{ \DynT }) }}$
%     \\
%     \pause
%     \vspace{0.5cm}
%     {\huge{$\longrightarrow^{*}$}}
%     \[
%       a \mapsto
%       \left\langle
%         \begin{array}{l}
%           {\coerced{\textcolor{RoyalPurple}{\coerced{(\lambda x. x)}{\FunT{\DynT}{\DynT} \cast \DynT}}}{\textcolor{Magenta}{\DynT \cast
%           (\FunT{\DynT}{\IntT})}}}, \\
%           {\coerced{\textcolor{RoyalPurple}{\coerced{a}{\RefT{(\PairT{\DynT}{\DynT})} \cast \DynT}}}{\textcolor{Magenta}{\DynT \cast
%           \RefT{(\PairT{(\FunT{\IntT}{\DynT})}{\DynT}})}}} 
%         \end{array}
%       \right\rangle
%     \]
%     \\
%     \pause
%     \vspace{0.5cm}
%     {\huge{$\longrightarrow^{*}$}}
%     \[
%       a \mapsto
%       \left\langle
%         \begin{array}{l}
%           \textcolor{RoyalPurple}{\coerced{(\lambda x. x)}{\FunT{\DynT}{\DynT} \cast \FunT{\DynT}{\IntT}}}, \\
%           \coerced{\textcolor{RoyalPurple}{\coerced{a}{\RefT{(\PairT{\DynT}{\DynT})} \cast \DynT}}}{\textcolor{Magenta}{\DynT \cast
%           \RefT{(\PairT{(\FunT{\IntT}{\DynT})}{\DynT})}}}
%         \end{array}
%       \right\rangle
%     \]
%     \\
%     \pause
%     \vspace{0.5cm}
%     Type of the heap cell should be $\PairT{ (\FunT{\IntT}{\IntT}) }{ \RefT{(\PairT{(\FunT{\IntT}{\DynT})}{\DynT})} }$
%   \end{center}
% \end{frame}

% \begin{frame}[fragile]{Old Work Writes Expressions to The Heap (Cont'd)}
%   \begin{center}
%     \[
%       a \mapsto
%       \left\langle
%         \begin{array}{l}
%           \textcolor{RoyalPurple}{\coerced{(\lambda x. x)}{\FunT{\DynT}{\DynT} \cast \FunT{\DynT}{\IntT}}}, \\
%           \coerced{\textcolor{RoyalPurple}{\coerced{a}{\RefT{(\PairT{\DynT}{\DynT})} \cast \DynT}}}{\textcolor{Magenta}{\DynT \cast
%           \RefT{(\PairT{(\FunT{\IntT}{\DynT})}{\DynT})}}}
%         \end{array}
%       \right\rangle
%     \]
%     \\
%     \vspace{0.5cm}
%     Type of the heap should be $\PairT{ (\FunT{\IntT}{\IntT}) }{
%       \RefT{(\PairT{(\FunT{\IntT}{\DynT})}{\DynT})} }$
%     \\
%     \pause
%     \vspace{0.5cm}
%     Eventually,
%     \[
%       a \mapsto
%       \textcolor{RoyalPurple}{\left\langle
%           \begin{array}{l}
%             \coerced{\coerced{(\lambda x. x)}{\FunT{\DynT}{\DynT} \cast \FunT{\DynT}{\IntT}}}{ \FunT{\DynT}{\IntT} \cast \FunT{\IntT}{\IntT}}, \\
%             a
%           \end{array}
%         \right\rangle}
%     \]
%   \end{center}
% \pause How could this behavior be implemented efficiently in a runtime system?
% \end{frame}

\section{Methodology}

\begin{frame}{Outline}
  \tableofcontents[currentsection]
\end{frame}

\begin{frame}{Outline}
  \begin{itemize}
  \item designed dynamic operational semantics for monotonic references
    that is space-efficient for functions and easy to implement in a
    compiler.
  \item implemented it as part of ahead-of-time compiler for gradually
    typed language.
  \item evaluated the performance on a suite of widely-used benchmarks.
    % \begin{itemize}
    %   \item measured the overhead of proxied references in static code.
    %   \item compared monotonic against Static Grift on static benchmarks
    %     to confirm the elimination of overheads in static code.
    %   \item compared monotonic against proxied on mixed typed code.
    %   \item compared coercions vs type-based casts on mixed typed code.
    %   \item compared Grift against dynamically typed languages on untyped code.
    %   \item compared Grift against statically typed languages on static code.
    % \end{itemize}
  \end{itemize}
\end{frame}

\begin{frame}{Answer to Q1: Normal-form Monotonic Coercion}
\begin{gather*}
  \inference{\AllTa \TypePreciseness \AllTa_2}{\RefCoercion{\AllTa} : \RefT{\AllTa_1} \Longrightarrow
    \RefT{\AllTa_2}}
\end{gather*}
\\
\[
    \begin{array}{rcl}
      \RefCoercion{\AllTa}\ \ComposeCoercion\ \RefCoercion{\AllTa'}
                                        & = &
                                              \RefCoercion{(\AllTa \glbt
                                              \AllTa')} \qquad \text{if } \AllTa \sim
                                              \AllTa'  \\
      \RefCoercion{\AllTa}\ \ComposeCoercion\ \RefCoercion{\AllTa'}
                                        & = &
                                              \FailCoercionSE
                                              \qquad\qquad\qquad \text{ if } \AllTa \not\sim
                                              \AllTa'
    \end{array}
  \]

\begin{prop}{Coercion composition height is bounded}
  \begin{center}
    $\| \seCa \ComposeCoercion \seCb\| \leq \max (\| \seCa \|, \| \seCb
    \|)$
  \end{center}
\end{prop}
\end{frame}

\begin{frame}[fragile]{Answer to Q2: Practical Dynamic Semantics}
  \begin{center}
% \begin{minted}
% [
% framesep=2mm,
% baselinestretch=1.2,
% bgcolor=lightgray,
% fontsize=\footnotesize,
% linenos,
% escapeinside=||
% ,mathescape=true
% ]
% {racket}
% (while (mref-cast-queue-not-empty?)
%           (let* ([address (mref-cast-queue-peek-address)]
%                  [t1 (Mbox-rtti-read address)]                 
%                  [t2 (mref-cast-queue-peek-type)]
%                  [t3 (greatest-lower-bound t1 t2)])
%             (mref-cast-queue-dequeue)
%             (when (not (= t1 t3))
%                    (let ([vi (Mbox-val-read address)]
%                          [cvi (apply-cast vi t1 t3 #f)])
%                      (Mbox-rtti-set! address t3)
%                      (Mbox-val-set! address cvi)))))
% \end{minted}
\begin{minted}
[
framesep=2mm,
fontsize=\footnotesize,
baselinestretch=1.2,
bgcolor=lightgray,
fontsize=\footnotesize,
linenos,
escapeinside=||
,mathescape=true
]
{C}
void apply_suspended_casts() {
    while (cast_queue_is_not_empty(ref_cq)) {
        int64_t * addr = cast_queue_peek_address(ref_cq);
        int64_t rtti1 = addr[0];
        int64_t ty = cast_queue_peek_type(ref_cq);
        int64_t rtti2 = greatest_lower_bound(rtti1, ty);
        cast_queue_dequeue(ref_cq);
        if (rtti1 != rtti2) {
            int64_t val1 = addr[1];
            int64_t val2 = apply_cast(val1, rtti1, rtti2, true);
            addr[0] = rtti2;
            addr[1] = val2;
        }
    }
}
\end{minted}
\end{center}
\end{frame}

% \begin{frame}[fragile]{Example of Reduction Using New Dynamic Semantics}
%   \begin{center}
%     The heap is
%     $a \mapsto \pair{\coerced{(\lambda x. x)}{\FunT{\DynT}{\DynT} \cast
%         \DynT}}{\coerced{a}{\RefT{(\PairT{\DynT}{\DynT})} \cast \DynT}}$
%     \\
%     \vspace{0.5cm}
%     $\coerced{\addr}{\PairT{(\FunT{\DynT}{\IntT})}{\RefT{(\PairT{(\FunT{\IntT}{\DynT})}{\DynT})}}}\longrightarrow
%     \ ?$
%     \\
%     \pause \vspace{0.5cm}

%     The heap becomes:
%     $\addr \mapsto \pair{\coerced{\lambda x. x}{\FunT{\DynT}{\DynT}\cast
%         \FunT{\DynT}{\IntT}}}{\addr}$
%     and the queue
%     \begin{tikzpicture}[
%       scale=0.45,
%       squarednode/.style={rectangle, draw=green!60, fill=green!5, very thick, minimum size=20mm},
%       ]
%       \node[squarednode] at (-1, 1) (cast){$(\addr, \PairT{(\FunT{\IntT}{\DynT})}{\DynT})$};
%     \end{tikzpicture}
%     \\
%     \pause
%     \vspace{0.5cm}
%     Eventually, the heap becomes:
%     $ \addr \mapsto\pair{\coerced{\lambda x. x }{\FunT{\DynT}{\DynT}
%         \cast\FunT{\IntT}{\IntT}}}{\addr}$
%   \end{center}
% \end{frame}

\section{Preliminary Performance Analysis}

\begin{frame}{Outline}
  \tableofcontents[currentsection]
\end{frame}

\frame{
  \frametitle{The Grift Compiler\footnote[frame]{\customcite{Kuhlenschmidt:2019aa}}}

  \begin{itemize}
  \item An ahead-of-time compiler written in Racket (\~20k LOC).
  \item The source language GTLC+ includes first-class functions,
    mutable arrays, recursive types, tuples, integers, and floats.
  \item Compiles GTLC+ to C and LLVM.
  \item Implements coercions.
  \item Values are 64 bits. % Values of type $\DynT$ are tagged.
  \item Specialize casts if source and target are not $\DynT$.
  \item Some optimization of function closures (e.g. direct calls).
  % \item No global optimizations, no type inference or specialization.
  \item Boehm garbage collector.
  \end{itemize}
}

% \begin{frame}{Grift\footnote[frame]{\customcite{Kuhlenschmidt:2019aa}}}
%   \begin{center}
%     \includegraphics{grift.jpg}
%   \end{center}
% \end{frame}

\begin{frame}{Answer to Q4(a): Grift vs Other Languages on Statically Typed Code}
  \begin{center}
    \adjincludegraphics[scale=.25,trim={0 0 0 0},clip]
                       {old_plots/extremes/Specialized_Lazy_Proxied_Coercions_static.png}
  \end{center}
\end{frame}

\begin{frame}{Answer to Q4(b): Grift vs Other Languages on Dynamically Typed Code}
  \begin{center}
    \adjincludegraphics[scale=.25,trim={0 0 0 0},clip]
                    {old_plots/extremes/Specialized_Lazy_Proxied_Coercions_dynamic.png}
  \end{center}
\end{frame}

\begin{frame}{Answer to Q3 \& Q4(c): Coercions vs Type-based Casts}
  \begin{center}
    \centering\adjincludegraphics[width=7cm, height=8.5cm,trim={0 0 0 {0.011\height}},clip]
                         {old_plots/fine/Proxied_Specialized/all/quicksort.png}
  \end{center}
\end{frame}

\begin{frame}{Answer to Q3 \& Q4(c): Coercions vs Type-based Casts (Cont'd)}
  \begin{center}
    \centering\adjincludegraphics[width=7cm, height=8.5cm,trim={0 0 0 {0.011\height}},clip]
                         {old_plots/fine/Proxied_Specialized/all/n_body.png}
  \end{center}
\end{frame}

\begin{frame}{Answer to Q3 \& Q4(c): Coercions vs Type-based Casts (Cont'd)}
  \begin{center}
    \centering\adjincludegraphics[width=7cm, height=8.5cm,trim={0 0 0 {0.011\height}},clip]
                         {old_plots/fine/Proxied_Specialized/all/ray.png}
  \end{center}
\end{frame}

\begin{frame}{Answer to Q5: Monotonic vs Proxied References on Statically-typed Code}
  \begin{center}
    \adjincludegraphics[scale=.25,trim={0 0 0 0},clip]
  {new_plots/extremes/Specialized_Lazy_Coercions_static_self.png}
  \end{center}
\end{frame}

\begin{frame}{Answer to Q6: Monotonic vs Proxied References on Mixed-typed Code}
  \begin{center}
    \includegraphics[scale=0.25]{new_plots/Specialized_UniformClosure_Coercions_Lazy/runtimes/matmult.png}
  \end{center}
\end{frame}

\begin{frame}{Answer to Q6: Monotonic vs Proxied References on Mixed-typed Code (Cont'd)}
  \begin{center}
    \includegraphics[scale=0.25]{new_plots/Specialized_UniformClosure_Coercions_Lazy/runtimes/blackscholes.png}
  \end{center}
\end{frame}

\begin{frame}{Answer to Q6: Monotonic vs Proxied References on Mixed-typed Code (Cont'd)}
  \begin{center}
    \adjincludegraphics[scale=.25,trim={0 0 0 0},clip]
  {new_plots/cumulative.png}
  \end{center}
\end{frame}

\begin{frame}{Conclusion}
  \begin{itemize}
  \item Proxies cause catastrophic slowdowns in mixed typed code.
  \item Coercions fixes the issue without introducing significant
    overheads in the average case.
  \item Proxies cause overheads in statically typed code regions.
  \item Monotonic references eliminate these overheads but it was not
    obvious how to implement prior work in a compiler and runtime
    system.
  \item To achieve efficiency in both statically typed and mixed typed
    code regions, I designed dynamic semantics that is straightforward
    to implement and uses coercions to guarantee space-efficiency.
  \item I implemented the new semantics in a compiler in gradually typed
    languages and showed that it is efficient.
  \end{itemize}
\end{frame}

\begin{frame}{Conclusion}
  \centering
  Mechanized Formalism: \url{https://github.com/deyaaeldeen/monotonic}
  Implementation in Grift: \url{https://github.com/Gradual-Typing/Grift}
  Benchmarks: \url{https://github.com/Gradual-Typing/benchmarks}
  Dynamizer: \url{https://github.com/Gradual-Typing/Dynamizer}
  \\
  \vspace{1cm}
  \centering
  \Huge Thank you!
\end{frame}

\end{document}

%%% Local Variables:
%%% mode: latex
%%% TeX-master: t
%%% End:
